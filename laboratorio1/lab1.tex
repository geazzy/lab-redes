\documentclass{article}
\usepackage[utf8]{inputenc}

\title{Relatório de Configuração de Rede e Firewall}
\author{Seu Nome}
\date{Data}

\begin{document}

\maketitle

\section{Introdução}
Este relatório descreve as configurações de rede e firewall implementadas nos roteadores Router1 e Router2. Também inclui uma explicação sobre o número de hosts permitidos em cada rede configurada.

\section{Configuração do Router1}
\begin{itemize}
    \item \textbf{Interfaces de Rede}:
    \begin{itemize}
        \item \textbf{enp0s3}: Conectado à Internet.
        \item \textbf{enp0s8}: Conectado à rede local extnet.
    \end{itemize}
    \item \textbf{Servidor DHCP}:
    \begin{itemize}
        \item Sub-rede: 10.254.254.0/30
        \item Quantidade de hosts: \(2^2 - 2 = 2\)
    \end{itemize}
\end{itemize}

\subsection{Script de Configuração do iptables}
\begin{verbatim}
#!/bin/bash

# Limpar as regras existentes
iptables -F
iptables -X

# Definir política padrão para as chains
iptables -P INPUT DROP
iptables -P FORWARD DROP
iptables -P OUTPUT ACCEPT

# Permitir tráfego na interface loopback
iptables -A INPUT -i lo -j ACCEPT
iptables -A OUTPUT -o lo -j ACCEPT

# Permitir tráfego já estabelecido e relacionado
iptables -A INPUT -m state --state ESTABLISHED,RELATED -j ACCEPT
iptables -A FORWARD -m state --state ESTABLISHED,RELATED -j ACCEPT

# Permitir SSH na interface enp0s3 e enp0s8
iptables -A INPUT -i enp0s3 -p tcp --dport 22 -j ACCEPT
iptables -A INPUT -i enp0s8 -p tcp --dport 22 -j ACCEPT

# Permitir ping (ICMP) em todas as interfaces
iptables -A INPUT -p icmp -j ACCEPT

# Permitir tráfego DHCP em todas as interfaces
iptables -A INPUT -p udp --sport 67:68 --dport 67:68 -j ACCEPT

# Encaminhar tráfego de rede local (extnet) para a Internet (enp0s3)
iptables -t nat -A POSTROUTING -o enp0s3 -j MASQUERADE

# Salvar as configurações
iptables-save > /etc/iptables/rules.v4

# Habilitar roteamento IPv4
echo 1 > /proc/sys/net/ipv4/ip_forward
\end{verbatim}

\subsubsection{Configuração do Servidor DHCP (dhcpd.conf)}
\begin{verbatim}
option domain-name "lab.net";
option domain-name-servers 1.1.1.1, 8.8.8.8;

default-lease-time 600;
max-lease-time 7200;

ddns-update-style none;
authoritative;

subnet 10.254.254.0 netmask 255.255.255.252 {
    range 10.254.254.1 10.254.254.2;
    option routers 10.254.254.1;
}
\end{verbatim}

\section{Configuração do Router2}
\begin{itemize}
    \item \textbf{Interfaces de Rede}:
    \begin{itemize}
        \item \textbf{enp0s8}: Conectado à rede extnet.
        \item \textbf{enp0s9}: Conectado à rede servnet.
        \item \textbf{enp0s10}: Conectado à rede intnet.
    \end{itemize}
    \item \textbf{Servidor DHCP}:
    \begin{itemize}
        \item Sub-rede servnet: 10.2.2.0/25
        \item Quantidade de hosts na servnet: \(2^7 - 2 = 126\)
        \item Sub-rede intnet: 10.3.2.0/23
        \item Quantidade de hosts na intnet: \(2^9 - 2 = 510\)
    \end{itemize}
\end{itemize}

\subsection{Script de Configuração do iptables}
\begin{verbatim}
#!/bin/bash

# Limpar as regras existentes
iptables -F
iptables -X

# Política padrão
iptables -P INPUT DROP
iptables -P FORWARD DROP
iptables -P OUTPUT ACCEPT

# Permitir tráfego de loopback
iptables -A INPUT -i lo -j ACCEPT
iptables -A OUTPUT -o lo -j ACCEPT

# Permitir tráfego já estabelecido e relacionado
iptables -A INPUT -m state --state ESTABLISHED,RELATED -j ACCEPT
iptables -A FORWARD -m state --state ESTABLISHED,RELATED -j ACCEPT

# Permitir tráfego de saída
iptables -A OUTPUT -j ACCEPT

# Permitir ping (ICMP) em todas as interfaces
iptables -A INPUT -p icmp -j ACCEPT
iptables -A OUTPUT -p icmp -j ACCEPT

# Permitir tráfego DHCP em todas as interfaces
iptables -A INPUT -p udp --sport 67:68 --dport 67:68 -j ACCEPT
iptables -A OUTPUT -p udp --sport 67:68 --dport 67:68 -j ACCEPT

# Encaminhar tráfego de rede local (intnet) para a Internet (enp0s8)
iptables -t nat -A POSTROUTING -s 10.3.2.0/23 -o enp0s8 -j MASQUERADE

# Encaminhar tráfego de rede local (servnet) para a Internet (enp0s8)
iptables -t nat -A POSTROUTING -s 10.2.2.0/25 -o enp0s8 -j MASQUERADE

# Permitir acesso de servidores na servnet aos clientes na intnet
iptables -A FORWARD -s 10.2.2.0/25 -d 10.3.2.0/23 -j ACCEPT

# Permitir acesso de clientes na intnet à servnet e à Internet
iptables -A FORWARD -s 10.3.2.0/23 -d 10.2.2.0/25 -j ACCEPT
iptables -A FORWARD -s 10.3.2.0/23 -o enp0s8 -j ACCEPT

# Permitir acesso de servidores na servnet a internet e aos clientes na intnet
iptables -A FORWARD -s 10.2.2.0/25 -d 10.3.2.0/23 -j ACCEPT
iptables -A FORWARD -s 10.2.2.0/25 -o enp0s8 -j ACCEPT

# Habilitar roteamento IPv4
echo 1 > /proc/sys/net/ipv4/ip_forward

# Salvar as regras
iptables-save > /etc/iptables/rules.v4
\end{verbatim}

\subsubsection{Configuração do Servidor DHCP (dhcpd.conf)}
\begin{verbatim}
option domain-name "lab.net";
option domain-name-servers 1.1.1.1;

default-lease-time 600;
max-lease-time 7200;

subnet 10.2.2.0 netmask 255.255.255.128 {
    interface enp0s9;
    range 10.2.2.2 10.2.2.126;
    option routers 10.2.2.1;
}

subnet 10.3.2.0 netmask 255.255.254.0 {
    interface enp0s10;
    range 10.3.2.2 10.3.3.254;
    option routers 10.3.2.1;
}
\end{verbatim}

\section{Verificação de Conectividade e Portas Abertas}

\subsection{Comandos \texttt{ss} para Verificar Portas Abertas}
Exemplo de comando \texttt{ss} para verificar portas abertas no Router1:
\begin{verbatim}
ss -tuln
\end{verbatim}

Exemplo de comando \texttt{ss} para verificar portas abertas no Router2:
\begin{verbatim}
ss -tuln
\end{verbatim}

\subsection{Comandos \texttt{ip} para Verificar Rotas}
Exemplo de comando \texttt{ip} para verificar rotas no Router1:
\begin{verbatim}
ip route show
\end{verbatim}

Exemplo de comando \texttt{ip} para verificar rotas no Router2:
\begin{verbatim}
ip route show
\end{verbatim}

\end{document}
